\documentclass{article}

\title{gitTraverse: Command-line tool for building document timelapse videos}

\author{Jos\'{e} Manuel Calder\'{o}n Trilla}

\date{February 14, 2013}

\begin{document}

\maketitle

\begin{abstract}
Using this tool will allow you to create timelapse videos of \LaTeX projects
that have been created using \verb=Git= for version control. The tool is
currently being developed and therefore still has some work to do. Currently,
the script will only create PDFs of the project at each commit and name them
according to the UNIX timestamp of the commit itself. 
\end{abstract}

\section{Introduction}
\verb=gitTraverse= is a poorly named Python script with the eventual goal of automating
the task of making timelapse videos for a document that has been made using \LaTeX and
\verb=Git=.

Currently, you run the script a \verb=.tex= file. The script will traverse your git 
history (from branch master) and create PDFs from the named \verb=.tex= file at
each commit. This is ugly but it's the first step towards our eventual goal.

The script will not crash if the \LaTeX build fails, it will just ignore that
commit and continue on with its traversal.

\end{document}
